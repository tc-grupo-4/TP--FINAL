\documentclass[a4paper,12pt,spanish]{report}

% Paquetes para Diagramas de Circuitos
\usepackage{tikz}
\usepackage[american, oldvoltagedirection]{circuitikz}

\usepackage{pgfplots}
\pgfplotsset{width=10cm, compat=1.9}

% Paquetes de Definicion de Idioma
\usepackage[utf8]{inputenc}	%Necesitamos los 8 bits, por eso se usa utf8 para el ASCII
\usepackage[spanish, es-tabla, es-nodecimaldot]{babel} %Por defecto las tablas se llaman cuadro en babel spanish, es-tabla es para definir el cuadro como tabla
\addto\shorthandsspanish{\spanishdeactivate{~<>}}
\renewcommand\spanishchaptername{Parte}

% Para ecuaciones 
\usepackage{relsize}
\usepackage{amsmath}
\usepackage{siunitx}


\usepackage{float}

\usepackage{amstext}
\usepackage{graphicx}
\usepackage{booktabs}
\usepackage{subcaption}

% Para utilizar el Índice para navegar en el documento
\usepackage{hyperref}	
\hypersetup{
    colorlinks=true,
    linkcolor=black,
    filecolor=magenta,      
    urlcolor=blue,
    citecolor=blue,    
}

\usepackage{bm}	%Es el bold map

% Para la carátula. Ver qué sacar
\usepackage[margin=0.5in]{geometry}
\usepackage{textcomp}
%%

%\usepackage{parskip}
\usepackage{fancyhdr}
\usepackage{vmargin}
\setmarginsrb{3 cm}{2.5 cm}{3 cm}{2.5 cm}{1 cm}{1.5 cm}{1 cm}{1.5 cm}
